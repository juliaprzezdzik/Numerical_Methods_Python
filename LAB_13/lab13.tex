\documentclass{article}
\usepackage{graphicx} 
\usepackage[utf8]{inputenc}
\usepackage[T1]{fontenc} 
\usepackage{tabularx} 
\usepackage{geometry} 
\usepackage{siunitx} 
\usepackage{caption}
\usepackage{amsmath} 
\usepackage{float}
\usepackage{array} 
\captionsetup[figure]{labelformat=empty} 
\usepackage{placeins}
\usepackage{xcolor}
\usepackage{multicol}
\usepackage{fancyhdr,lastpage}
\usepackage{fancyvrb}
\usepackage{titling}
\usepackage{titlesec}
\usepackage{amsmath}
\usepackage{booktabs}
\usepackage{amssymb}
\usepackage{enumitem}
\usepackage{booktabs}
\usepackage{tikz}
\pagestyle{fancy}
\newgeometry{tmargin=2cm, bmargin=2cm, lmargin=2cm, rmargin=2cm}
\Large 

\begin{document}
\Large

\title{\LARGE Sprawozdanie nr 13 z przedmiotu Metody Numeryczne\\
       \LARGE Całkowanie numeryczne przy użyciu kwadratur Gaussa}
\author{Julia Przeździk}
\date{18 czerwca 2024 r.}
\maketitle

\rhead{\small{Całkowanie numeryczne przy użyciu kwadratur Gaussa}}
\lhead{\small{Metody Numeryczne}}
\cfoot{Strona \thepage\ z \pageref{LastPage}} 
\large

\section{Cel ćwiczenia}
Zadanie polegało na wyliczeniu właściwych i niewłaściwych całek w celu zapoznania się z ideą  kwadratur Gaussa i możliwością ich wykorzystania w całkowaniu numerycznym.

\section{Opis problemu}

Na początku należało obliczyć numerycznie przy użyciu kwadratury Gaussa-Legendre'a wartość całki
\begin{equation*}
c_1 = \int_{0}^{2} \frac{x}{4x^2 + 1} dx
\end{equation*}
\noindent
Wartość dokładną można było obliczyć korzystając z rozwiązania analitycznego
\begin{equation*}
c_{1,a} = \int \frac{x}{a^2 x^2 \pm c^2} dx = \frac{1}{2a^2} \ln|a^2 x^2 \pm c^2|
\end{equation*}

\noindent
Następnie należało wykonać wykres $|c_1 - c_{1,a}| = f(n)$, dla liczby węzłów $n = 2, 3, \ldots, 20$.
\noindent
Kolejne zadanie polegało na obliczeniu numerycznie przy użyciu kwadratury Gaussa-Laguerre'a wartości całki
\begin{equation*}
c_2 = \int_{0}^{\infty} x^k \exp(-x) dx
\end{equation*}

\noindent
Wartość dokładną można było obliczyć korzystając z rozwiązania analitycznego
\begin{equation*}
c_{2,a} = \int_{0}^{\infty} x^k \exp(-x) dx = k!
\end{equation*}

\noindent
Następnie należało wykonać wykresy: $|c_2 - c_{2,a}| = f(n)$, dla $k = 5$ i dla liczby węzłów $n = 2, 3, \ldots, 20$ oraz  $|c_2 - c_{2,a}| = f(n)$, dla $k = 10$ i dla liczby węzłów $n = 2, 3, \ldots, 20$.
\noindent
Kolejne ćwiczenie polegało na obliczneiu numerycznie przy użyciu kwadratury Gaussa-Hermite'a wartości podwójnej całki
\begin{equation*}
c_3 = \int_{-\infty}^{\infty} \int_{-\infty}^{\infty} \sin^2(x) \sin^4(y) \exp(-x^2 - y^2) dx dy \quad (c_{\text{dok}} = 0.1919832644)
\end{equation*}

\noindent
Należało również wykonać wykonać wykres $|c_3 - c_{\text{dok}}| = f(n)$, dla liczby węzłów $n = 2, 3, \ldots, 15$.

\section{Część teoretyczna}

Kwadratury Gaussa to metody \textit{całkowania numerycznego} polegające na takim wyborze wag $w_1, w_2, \ldots, w_n$ i węzłów interpolacji $t_1, t_2, \ldots, t_n \in [a, b]$ aby wyrażenie
\begin{equation*}
\sum_{i=1}^{n} w_i f(t_i)
\end{equation*}
najlepiej przybliżało całkę
\begin{equation*}
I(f) = \int_{a}^{b} w(x)f(x)dx,
\end{equation*}
gdzie $f$ jest dowolną funkcją określoną na odcinku $[a, b]$, a $w$ jest tzw. funkcją wagową spełniającą warunki
\begin{enumerate}
    \item $w(x) > 0$,
    \item $\forall_{k \in \mathbb{N}} \int_{a}^{b} x^k w(x) dx$ jest skończona,
    \item Jeżeli $p$ jest wielomianem takim, że $\forall_{x \in [a,b]} p(x) \geq 0$, to jeśli $\int_{a}^{b} w(x)p(x)dx = 0$, wtedy $p \equiv 0$.
\end{enumerate}

\noindent
Określmy iloczyn skalarny z wagą
\begin{equation*}
\langle f, g \rangle_{w} = \int_{a}^{b} w(x) f(x) g(x) dx.
\end{equation*}

\noindent
Dwa wielomiany są ortogonalne względem tego iloczynu skalarnego, jeśli $\langle f, g \rangle_{w} = 0$.
\noindent
Wszystkie kwadratury Gaussa wywodzą się z twierdzenia udowodnionego przez niego:

\begin{enumerate}
    \item Jeżeli $t_1, t_2, \ldots, t_n \in [a, b]$ są pierwiastkami $n$-tego wielomianu ortogonalnego $p_n(x)$ oraz $w_1, w_2, \ldots, w_n$ są rozwiązaniami układu równań:
    \begin{equation*}
    \begin{cases}
        p_0(t_1) w_1 + \ldots + p_0(t_n) w_n &= \langle p_0, p_0 \rangle_{w} \\
        p_1(t_1) w_1 + \ldots + p_1(t_n) w_n &= 0 \\
        \vdots \\
        p_{n-1}(t_1) w_1 + \ldots + p_{n-1}(t_n) w_n &= 0
    \end{cases}
    \end{equation*}
\noindent
    dla każdego wielomianu $p$ stopnia nie większego niż $2n - 1$ zachodzi
    \begin{equation*}
    \int_{a}^{b} w(x)p(x)dx = \sum_{i=1}^{n} w_i p(t_i). \quad (*)
    \end{equation*}
    
    Ponadto $w_i > 0$.
    
    \item Jeżeli dla pewnego ciągu węzłów $x_1, x_2, \ldots, x_n \in [a, b]$ oraz ciągu wag $v_1, v_2, \ldots, v_n$ dla dowolnego wielomianu $p$ stopnia nie większego niż $2n - 1$ zachodzi warunek $(*)$, to $x_i = t_i$ oraz $v_i = w_i$ z dokładnością do kolejności.
    
    \item Dla dowolnego ciągu węzłów $x_1, x_2, \ldots, x_n \in [a, b]$ oraz ciągu wag $v_1, v_2, \ldots, v_n$ istnieje wielomian stopnia $2n$, dla którego nie zachodzi warunek $(*)$.
\end{enumerate}

\noindent
Kwadratury z \textit{przedziału} $[-1, 1]$ z wagą $w \equiv 1$ to kwadratury Gaussa-Legendre'a
\begin{equation*}
I(f) = \int_{-1}^{1} f(x) dx \approx \sum_{i=1}^{n} w_i f(t_i),
\end{equation*}
gdzie $t_i$ to \textit{pierwiastki} n-tego \textit{wielomianu Legendre'a}.

\noindent
Kwadratury z wagą $w(x) = e^{-x}$ nazywamy kwadraturami Gaussa-Laguerre'a
\begin{equation*}
I(f) = \int_{0}^{\infty} e^{-x} f(x) dx \approx \sum_{i=1}^{n} w_i f(t_i),
\end{equation*}
gdzie $t_i$ to pierwiastki n-tego \textit{wielomianu Laguerre'a}.
\noindent
Kwadratury z wagą $w(x) = e^{-x^2}$ nazywamy kwadraturami Gaussa-Hermite'a
\begin{equation*}
I(f) = \int_{-\infty}^{\infty} e^{-x^2} f(x) dx \approx \sum_{i=1}^{n} w_i f(t_i),
\end{equation*}
gdzie $t_i$ to pierwiastki n-tego \textit{wielomianu Hermite'a}.

\section{Wykorzystanie metody i otrzymane wyniki} 

\begin{table}[h!]
\centering
\begin{tabular}{ccc}
\toprule
Liczba węzłów (n) & Błąd bezwzględny \(|c_3 - c_{\text{dok}}|\) \\
\midrule
2  & 0.10511129658668245  \\
3  & 0.1799388875299586   \\
4  & 0.16422144413869394  \\
5  & 0.16217769532040988  \\
6  & 0.1694836641002307   \\
7  & 0.16103332931392078  \\
8  & 0.1672299192989416   \\
9  & 0.16293873745843435  \\
10 & 0.16558650437443537  \\
11 & 0.16401669137696012  \\
12 & 0.1649038019866156   \\
13 & 0.16441691842310513  \\
14 & 0.16467692142475499  \\
15 & 0.1645409962401517   \\
\bottomrule
\end{tabular}
\caption{Błąd bezwzględny w kwadraturze Gaussa-Legendre'a}
\label{tab:error}
\end{table}

\begin{figure}[H]
    \centering
    \includegraphics[width=0.9\linewidth]{Zrzut ekranu 2024-06-18 o 22.13.07.png}
    \caption*{}
    \label{fig:enter-label}
\end{figure}

\begin{figure}[H]
    \centering
    \includegraphics[width=0.9\linewidth]{Zrzut ekranu 2024-06-18 o 22.13.53.png}
    \caption*{}
    \label{fig:enter-label}
\end{figure}

\begin{table}[h!]
\centering
\begin{tabular}{cc}
\toprule
Liczba węzłów (n) & Suma współczynników (k=5) \\
\midrule
2  & 1.0000000000000000 \\
3  & 1.0000000000000000 \\
4  & 1.0000000000000002 \\
5  & 1.0000000000000000 \\
6  & 0.9999999999999997 \\
7  & 1.0000000000000000 \\
8  & 0.9999999999999998 \\
9  & 1.0000000000000000 \\
10 & 1.0000000000000000 \\
11 & 1.0000000000000000 \\
12 & 0.9999999999999996 \\
13 & 0.9999999999999999 \\
14 & 1.0000000000000002 \\
15 & 1.0000000000000000 \\
16 & 0.9999999999999999 \\
17 & 0.9999999999999999 \\
18 & 1.0000000000000000 \\
19 & 1.0000000000000004 \\
20 & 0.9999999999999998 \\
\bottomrule
\end{tabular}
\caption{Suma współczynników w kwadraturze Gaussa-Laguerre'a dla k=5}
\label{tab:coeff_k5}
\end{table}

\begin{table}[h!]
\centering
\begin{tabular}{cc}
\toprule
Liczba węzłów (n) & Suma współczynników (k=10) \\
\midrule
2  & 1.0000000000000000 \\
3  & 1.0000000000000000 \\
4  & 1.0000000000000002 \\
5  & 1.0000000000000000 \\
6  & 0.9999999999999997 \\
7  & 1.0000000000000000 \\
8  & 0.9999999999999998 \\
9  & 1.0000000000000000 \\
10 & 1.0000000000000000 \\
11 & 1.0000000000000000 \\
12 & 0.9999999999999996 \\
13 & 0.9999999999999999 \\
14 & 1.0000000000000002 \\
15 & 1.0000000000000000 \\
16 & 0.9999999999999999 \\
17 & 0.9999999999999999 \\
18 & 1.0000000000000000 \\
19 & 1.0000000000000004 \\
20 & 0.9999999999999998 \\
\bottomrule
\end{tabular}
\caption{Suma współczynników w kwadraturze Gaussa-Laguerre'a dla k=10}
\label{tab:coeff_k10}
\end{table}

\clearpage
\newpage
\begin{figure}[H]
    \centering
    \includegraphics[width=0.9\linewidth]{Zrzut ekranu 2024-06-18 o 22.35.22.png}
    \caption*{}
    \label{fig:enter-label}
\end{figure}

\begin{figure}[H]
    \centering
    \includegraphics[width=0.9\linewidth]{Zrzut ekranu 2024-06-18 o 22.36.25.png}
    \caption*{}
    \label{fig:enter-label}
\end{figure}

\begin{figure}[H]
    \centering
    \includegraphics[width=0.9\linewidth]{Zrzut ekranu 2024-06-18 o 22.37.00.png}
    \caption*{}
    \label{fig:enter-label}
\end{figure}

\begin{figure}[H]
    \centering
    \includegraphics[width=0.9\linewidth]{Zrzut ekranu 2024-06-18 o 22.37.29.png}
    \caption*{Enter Caption}
    \label{fig:enter-label}
\end{figure}

\begin{figure}[H]
    \centering
    \includegraphics[width=1.0\linewidth]{Zrzut ekranu 2024-06-18 o 23.04.32.png}
    \caption*{}
    \label{fig:enter-label}
\end{figure}

\section{Wnioski}
Celem ćwiczenia było zrozumienie i zastosowanie kwadratur Gaussa do obliczania całek numerycznych.\\
\noindent
Pierwsze zadanie wykazało, że kwadratura Gaussa-Legendre'a jest skuteczna w przybliżaniu całek na skończonym przedziale 
[0,2]. Wzrost liczby węzłów 
n prowadził do zmniejszenia błędu, co potwierdza dokładność tej metody.\\
\noindent
Drugie zadanie dotyczyło kwadratury Gaussa-Laguerre'a dla przedziału 
[0,$\infty$] i różnych wartości 
k. Metoda okazała się precyzyjna, a błędy zmniejszały się wraz ze wzrostem liczby węzłów, co potwierdziło jej skuteczność. \\
\noindent
Ostatnie zadanie, wykorzystujące kwadraturę Gaussa-Hermite'a do obliczenia podwójnej całki, również wykazało dużą dokładność metody. Błąd bezwzględny szybko malał do wartości bliskiej zeru przy odpowiedniej liczbie węzłów. \\
\noindent
Podsumowując, wszystkie trzy metody kwadratur Gaussa były bardzo skuteczne, a ich wszechstronność i dokładność potwierdzają przydatność w różnych zastosowaniach numerycznych.

\end{document}

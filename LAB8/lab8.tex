\documentclass{article}
\usepackage{graphicx} 
\usepackage[utf8]{inputenc}
\usepackage[T1]{fontenc} 
\usepackage{tabularx} 
\usepackage{geometry} 
\usepackage{siunitx} 
\usepackage{caption}
\usepackage{amsmath} 
\usepackage{float}
\usepackage{array} 
\captionsetup[figure]{labelformat=empty} 
\usepackage{placeins}
\usepackage{xcolor}
\usepackage{multicol}
\usepackage{fancyhdr,lastpage}
\usepackage{fancyvrb}
\usepackage{titling}
\usepackage{titlesec}
\usepackage{amssymb}
\usepackage{enumitem}
\usepackage{tikz}
\pagestyle{fancy}
\newgeometry{tmargin=2cm, bmargin=2cm, lmargin=2cm, rmargin=2cm}
\Large %wielkosc czcionki

\begin{document}
\Large

\title{\LARGE Sprawozdanie nr 9 z przedmiotu Metody Numeryczne\\
       \LARGE Aproksymacja funkcji okresowych}
\author{Julia Przeździk}
\date{18 maja 2024 r.}
\maketitle

\rhead{\small{Aproksymacja funkcji okresowych}}
\lhead{\small{Metody Numeryczne}}
\cfoot{Strona \thepage\ z \pageref{LastPage}} 
\large

\section{Cel ćwiczenia}

Zadanie polegało na aproksymacji trzech funkcji okresowych $f_1, f_2, f_3$, czyli wyznaczeniu współczynników kombinacji liniowej za pomocą odpowiednich wzorów i z uwzględnieniem określonego przedziału aproksymacji. Należało napisać program przeprowadzający tę operację w wybranym języku programowania, a następnie narysować wykresy odpowiednich funkcji.

\section{Opis problemu}
Zdefiniowano 3 funkcje: 

\begin{equation*}
\begin{aligned}
f_1(x) &= 2\sin(x) + \sin(2x) + 2\sin(3x) + \alpha \\
f_2(x) &= 2\sin(x) + \sin(2x) + 2\cos(x) + \cos(2x) \\
f_3(x) &= 2\sin(1.1x) + \sin(2.1x) + 2\sin(3.1x)
\end{aligned}
\end{equation*}

\noindent
gdzie $\alpha = \frac{\text{rand()}}{\text{RAND\_MAX} + 1.0} - 0.5$. Zadanie polegało na aproksymacji $f_1, f_2, f_3$ za pomoca funkcji $F(x)$:

\begin{equation*}
F(x) = \sum_{k=0}^{M_s} a_k \sin(kx) + \sum_{j=0}^{M_c} b_j \cos(jx)
\end{equation*}

\noindent
Należało przyjąć liczbę węzłów $2n = 100$ oraz przedział aproksymacji $x \in [0, 2\pi)$, a także:
\begin{itemize}
    \item dla $f_1$: $\alpha = 0$, $(M_s, M_c) = {(5,5)}$
    \item dla $f_2$: $(M_s, M_c) = {(5,5)}$
    \item dla $f_2$: $(M_s, M_c) = {(5,0),(5,5), (10, 10)}$
    \item ponownie dla $f_1$, dla każdego węzła należało obliczyć wartość $\alpha$ z podanego powyżej wzoru,; $(M_s, M_c) = {(5,5),(30,30)}$
\end{itemize}

\noindent
Dla powyższych funkcji należało wyznaczyć współczynniki $a_k$ oraz $b_j$, a także narysować wykresy $f_i(x)$ i $F_i(x)$ na jednym rysunku.

\section{Część teoretyczna}

\subsection{Aproksymacja funkcji trygonometrycznych}

 Funkcje okresowe aproksymuje się przy użyciu funkcji trygonometrycznych, czyli w bazie
    \[
    1, \sin(x), \cos(x), \sin(2x), \cos(2x), \ldots
    \]
\noindent
Wielomian trygonometryczny o okresie \(2\pi\) ma postać:
    \[
    Q_m(x) = \frac{a_0}{2} + \sum_{j=1}^m \left[ a_j \cos(jx) + b_j \sin(jx) \right]
    \]

\noindent
Jeśli funkcja \(f(x)\) jest określona na dyskretnym zbiorze równoodległych punktów, a liczba punktów jest parzysta i wynosi \(2n\):
    \[
    x_i = \frac{\pi i}{n}, \quad i = 0, 1, 2, \ldots, 2n-1
    \]

    \begin{equation*}
    \sum_{i=0}^{2N-1} \sin(mx_i) \sin(kx_i) =
    \begin{cases} 
    0, & m \neq k \\
    N, & m = k \neq 0 \\
    0, & m = k = 0 
    \end{cases}
    \end{equation*}

    \begin{equation*}
    \sum_{i=0}^{2N-1} \cos(mx_i) \cos(kx_i) =
    \begin{cases} 
    0, & m \neq k \\
    N, & m = k \neq 0 \\
    2N, & m = k = 0 
    \end{cases}
    \end{equation*}

    \[
    \sum_{i=0}^{2N-1} \cos(mx_i) \sin(kx_i) = 0, \quad \text{m, k - dowolne}
    \]

\noindent
Poszukiwany wielomian jest postaci:
\[
F(x) = \frac{1}{2} a_0 + \sum_{j=1}^{m} \left[ a_j \cos(jx) + b_j \sin(jx) \right], \quad m < N
\]

\noindent
Współczynniki \(a_j\) oraz \(b_j\) wyznacza się z warunku minimalizacji wyrażenia:
\[
\sum_{i=0}^{2N-1} \left[ f(x_i) - F(x_i) \right]^2 = \min
\]
\noindent
co prowadzi do zależności na współczynniki:
\[
a_0 = \frac{1}{2N} \sum_{i=0}^{2N-1} f(x_i)
\]

\[
a_{j>0} = \frac{1}{N} \sum_{i=0}^{2N-1} f(x_i) \cos(jx_i) = \frac{1}{N} \sum_{i=0}^{2N-1} f(x_i) \cos\left( \frac{\pi i j}{N} \right)
\]

\[
b_j = \frac{1}{N} \sum_{i=0}^{2N-1} f(x_i) \sin(jx_i) = \frac{1}{N} \sum_{i=0}^{2N-1} f(x_i) \sin\left( \frac{\pi i j}{N} \right)
\]

\section{Wykorzystanie metody oraz otrzymane wyniki}
Przy pomocy języką Python oraz biblioteki NumPy stworzono funkcje oraz wygenerowano poniższe wykresy:

\begin{figure}[H]
    \centering
    \includegraphics[width=0.8\linewidth]{Zrzut ekranu 2024-05-18 o 23.48.52.png}
    \caption{\textbf{Rysunek nr 1: }Aproksymacja funkcji $f1(x)$}
    \label{fig:enter-label}
\end{figure}

\begin{figure}[H]
    \centering
    \includegraphics[width=0.8\linewidth]{Zrzut ekranu 2024-05-18 o 23.49.28.png}
    \caption{\textbf{Rysunek nr 2: }Aproksymacja funkcji $f2(x)$}
    \label{fig:enter-label}
\end{figure}

\begin{figure}[H]
    \centering
    \includegraphics[width=0.8\linewidth]{Zrzut ekranu 2024-05-18 o 23.50.01.png}
    \caption{\textbf{Rysunek nr 3: }Aproksymacja funkcji $f3(x)$}
    \label{fig:enter-label}
\end{figure}

\begin{figure}[H]
    \centering
    \includegraphics[width=0.8\linewidth]{Zrzut ekranu 2024-05-18 o 23.50.39.png}
    \caption{\textbf{Rysunek nr 4: }Aproksymacja funkcji $f3(x)$}
    \label{fig:enter-label}
\end{figure}

\begin{figure}[H]
    \centering
    \includegraphics[width=0.8\linewidth]{Zrzut ekranu 2024-05-18 o 23.52.27.png}
    \caption{\textbf{Rysunek nr 5: }Aproksymacja funkcji $f3(x)$}
    \label{fig:enter-label}
\end{figure}

\begin{figure}[H]
    \centering
    \includegraphics[width=0.8\linewidth]{Zrzut ekranu 2024-05-18 o 23.53.14.png}
    \caption{\textbf{Rysunek nr 6: }Aproksymacja funkcji $f1(x)$}
    \label{fig:enter-label}
\end{figure}

\begin{figure}[H]
    \centering
    \includegraphics[width=0.5\linewidth]{Zrzut ekranu 2024-05-23 o 15.55.44.png}
    \caption{\textbf{Rysunek nr 7} :Współczynniki dla Ms = Mc = 5}
    \label{fig:enter-label}
\end{figure}

\begin{figure}[H]
    \centering
    \includegraphics[width=0.5\linewidth]{Zrzut ekranu 2024-05-23 o 15.56.16.png}
    \caption{\textbf{Rysunek nr 8} :Współczynniki dla Ms = Mc = 30}
    \label{fig:enter-label}
\end{figure}

\newpage
\section{Wnioski}

Na podstawie przeprowadzonej aproksymacji funkcji $f_1(x)$, $f_2(x)$ oraz $f_3(x)$ można wyciągnąć kilka istotnych wniosków.\\
Przede wszystkim, aproksymacja funkcji okresowych za pomocą funkcji trygonometrycznych okazała się skuteczna. Wykresy funkcji aproksymowanych $F(x)$ w porównaniu do funkcji oryginalnych $f(x)$ pokazują, że przy odpowiednio dobranych wartościach $M_s$ oraz $M_c$, aproksymowane funkcje dobrze odwzorowują kształt i zachowanie funkcji pierwotnych.\\
Dla funkcji $f_1(x)$, w przypadku gdy $\alpha = 0$ oraz $(M_s, M_c) = (5, 5)$, aproksymacja była dokładna, co jest widoczne na wykresach. Oznacza to, że przy takim doborze parametrów można uzyskać dokładne odwzorowanie funkcji.\\
Dla funkcji $f_2(x)$, przy tych samych wartościach $(M_s, M_c) = (5, 5)$, również uzyskano dobrą aproksymację. Wskazuje to, że metoda jest uniwersalna i może być stosowana do różnych funkcji okresowych.\\
W przypadku funkcji $f_3(x)$, gdzie rozpatrywano różne wartości $(M_s, M_c)$, wyniki wskazują, że zwiększenie liczby współczynników poprawia jakość aproksymacji. Szczególnie dla wartości $(M_s, M_c) = (10, 10)$, aproksymacja była bardziej dokładna w porównaniu do $(M_s, M_c) = (5, 0)$ czy $(M_s, M_c) = (5, 5)$. To pokazuje, że dla bardziej złożonych funkcji konieczne może być zwiększenie liczby współczynników, aby uzyskać lepsze przybliżenie.\\
Dodatkowo, aproksymacja funkcji $f_1(x)$ dla różnych losowych wartości $\alpha$ z parametrami $(M_s, M_c) = (30, 30)$ pokazała, że metoda jest stabilna i skuteczna nawet przy zmiennych wartościach w funkcji. Większa liczba współczynników pozwala na lepsze dopasowanie się do funkcji o większej zmienności.\\
Podsumowując, metoda aproksymacji funkcji okresowych za pomocą funkcji trygonometrycznych jest efektywna i wszechstronna. Dobór odpowiednich parametrów $(M_s, M_c)$ jest kluczowy dla uzyskania dokładnej aproksymacji, a zwiększenie liczby współczynników może znacząco poprawić jakość przybliżenia. Metoda ta jest szczególnie przydatna dla funkcji o wysokiej zmienności i złożoności.

\end{document}
